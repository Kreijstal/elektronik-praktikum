%
\section{Einführung}

In diesem Dokument werden grundlegende Konzepte und Anwendungen von aktiven elektronischen Filtern untersucht. Elektronische Filter sind unverzichtbare Bestandteile in modernen elektronischen Schaltungen und Systemen. Sie ermöglichen es, unerwünschte Frequenzen aus einem Signal herauszufiltern oder gewünschte Frequenzbereiche hervorzuheben. Das Verständnis ihrer Funktionsweise und Anwendung ist daher von entscheidender Bedeutung für jeden, der sich mit Elektronik und Schaltungsentwurf beschäftigt.

Im weiteren Verlauf werden verschiedene Typen von Filtern wie den Integrator, Hochpass, Tiefpass und Bandpass analysiert. Dazu gehören die theoretische Grundlage, der Schaltplan, der Aufbau im Labor und die Ergebnisse aus Simulationen und Messungen.

% Nächstes Protokoll https://www.overleaf.com/5889796663xvtntgfxcxwg#fb9cc4
