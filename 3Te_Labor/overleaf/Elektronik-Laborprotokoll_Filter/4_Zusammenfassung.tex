%
%
%
\section{Zusammenfassung}
%

%Die im Labor aufgenommenen Messwerte und die durch KiCad erstellten
%Simulationswerte für verschiedene Filtertypen bieten eine umfassende

Im Rahmen dieses Labortermins wurden verschidene Filterarten untersucht. Eine Integrator-Schaltung, Tiefpassfilter erster Ordnung, PI Flter, Sallen-Key Filter sind in der Fritzing-Umgebung simuliert und auf dem Steckbrett aufgebaut. Alle Schaltungen, die auf dem Steckbrett aufgebaut wurden, wurden aber auch mithilfe KiCads simuliert. Ein Universalfilter, ein Allpassfilter und ein Bessel Tiefpass dritter Ordnung mit Lag Kompensator wurden zwar nicht im Labor aufgebaut aber in der KiCad-Umgebung simuliert. Die gemessenen Werte hatten meistens eine sehr hohe Korrelaiton mit den simulierten Werten. An einigen Messungen war zusätlich ein Überschwingengen und ein Hochpass-Charakter zu bemerken, welche Abweichungen aus der Theorie verursacht hatte und in die Eigenschaften des verwendeten Operationsverstärker zurückzuführen war. Im allgemeinen wurden in der Auswertung auf die grundlegenden Eigenschaften jeweiliger Filterarten eingegangen und mithilfe der Ergebnisse aus den Messungen  und der Simulation wurden die theoretischen Überlegungen über die untersuchten Filter bestätigt.

 


