Analyse der Filtercharakteristiken und deren Verhalten in realen
Anwendungsszenarien. Die Gegenüberstellung von Theorie, Simulation und
Messung ermöglicht eine detaillierte Untersuchung der Filterperformance
und -eigenschaften.

\subsubsection{Integrator}\label{integrator}

Der invertierende Integrator zeigt eine erhöhte Amplitudenverstärkung
bei niedrigeren Frequenzen. Dies bestätigt das theoretische Verhalten
eines Integrators, dessen Verstärkung gegen unendlich geht, wenn sich
die Frequenz Null nähert. Die experimentellen und simulierten Ergebnisse
stimmen gut überein, mit kleineren Abweichungen bei hohen Frequenzen.

\subsubsection{Aktive Filter Erster
Ordnung}\label{aktive-filter-erster-ordnung}

Das aktive Tiefpassfilter erster Ordnung dämpft höhere Frequenzanteile
und lässt niedrigere Frequenzen unverändert passieren. Die
Grenzfrequenz, definiert durch die Filterkomponenten, wurde bei etwa 159
Hz beobachtet. Sowohl Simulation als auch Messung zeigen ein
konsistentes Verhalten mit der Theorie.

\subsubsection{PI-Filter}\label{pi-filter}

Das PI-Filter verbindet die Eigenschaften eines Tiefpasses mit denen
eines Integrators, was zu einer spezifischen Übertragungsfunktion und
einem charakteristischen Frequenzverhalten führt. Die Ergebnisse weisen
auf eine effektive Kombination der beiden Filtertypen hin.

\subsubsection{Sallen-Key und MFB}\label{sallen-key-und-mfb}

Der Sallen-Key-Filter zeigte in der Messung und Simulation eine
konsistente Performance, mit kleinen Abweichungen bei höheren
Frequenzen. Diese Abweichungen könnten auf nicht-ideale Komponenten oder
parasitäre Effekte zurückzuführen sein.

\subsubsection{Universalfilter}\label{universalfilter}

Die Simulation des Universalfilters zeigt deutlich den Unterschied in
der Flankensteilheit zwischen Bandpass, Hochpass und Tiefpass. Die
Phasenantwort legt nahe, dass Hochpass- und Tiefpassfilter invertierend
sind.

\subsubsection{Allpass und
Phasenkompensatoren}\label{allpass-und-phasenkompensatoren}

Die Analyse des Bessel-Tiefpasses und des Lag-Kompensators zeigt, dass
diese Kombination zu einer Verschiebung der Eckfrequenz und einer
Veränderung der Phasencharakteristik führt. Die Übertragungsfunktion und
das Phasen-/Amplitudenverhalten bestätigen die typischen Eigenschaften
dieser Filtertypen.

Insgesamt zeigen die experimentellen und simulierten Ergebnisse eine
gute Übereinstimmung mit der theoretischen Erwartung. Kleine
Abweichungen bei höheren Frequenzen können auf Messungenauigkeiten oder
nicht-ideale Bedingungen in der experimentellen Aufstellung
zurückgeführt werden. Diese Analysen bieten wertvolle Einblicke in das
Verhalten und die Eigenschaften verschiedener Filtertypen und deren
Anwendbarkeit in praktischen Szenarien.