%
%
%
\section{Zusammenfassung}
%

Im Rahmen dieses Labortermins wurden verschiedene Oszillatorarten untersucht, darunter der Phasenschiebeoszillator, die astabile Kippstufe, das Monoflop und der Dreieck-Rechteck-Oszillator. Diese Schaltungen wurden sowohl in der Fritzing-Umgebung simuliert als auch auf dem Steckbrett aufgebaut. Zusätzlich zu den praktischen Aufbauten erfolgte die Simulation aller Schaltungen mithilfe von KiCad.

Die Simulationen in KiCad und LTSpice wiesen bedeutende Ähnlichkeiten auf. Dennoch traten Unterschiede aufgrund der internen Simulation des NE555-Schaltkreises sowie des Phasenschiebeoszillator-Simulators und ähnlicher Werkzeuge auf. Im Allgemeinen stimmen die Mess- und Simulationsergebnisse gut überein und bestätigen die Erwartungen, die auf den theoretischen Überlegungen basierten.