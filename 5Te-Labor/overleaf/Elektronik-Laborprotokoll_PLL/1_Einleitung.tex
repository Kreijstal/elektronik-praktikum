\section{Einleitung}

Die Phasenregelschleife (PLL) stellt eine hochwirksame Regelungseinrichtung dar, mit deren Hilfe die Phasenlage eines Signals präzise gesteuert werden kann. Ihr Hauptziel besteht darin, die Phasenabweichung zu einem Referenzsignal konstant zu halten, wodurch die Phase und Frequenz des Signals in Echtzeit synchronisiert werden können. In diesem Protokoll werden die einzelnen Funktionsblöcke einer PLL, einschließlich des Phasendetektors, des Schleifenfilters und des spannungsgesteuerten Oszillators (VCO), im Detail dimensioniert und analysiert. Die hybride PLL, die hier implementiert wird, nutzt einen VCO mit analoger Steuerspannung für eine genauere Regelung der Phasen- und Frequenzeigenschaften. Jeder Funktionsblock wird gründlich analysiert, um ein klares Verständnis der Struktur und Funktionsweise der PLL zu vermitteln. Das Protokoll bietet somit einen eingehenden Einblick in die Implementierung und Analyse einer hybriden PLL, wobei die grundlegenden Prinzipien der Phasenregelschleife beleuchtet werden.