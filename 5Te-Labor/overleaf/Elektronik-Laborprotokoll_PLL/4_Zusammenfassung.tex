\section{Zusammenfassung}

Im Rahmen des PLL-Versuchs wurden mehrere entscheidende Schritte durchgeführt. Zunächst erfolgte die Dimensionierung des Schmitt-Triggers sowie des Integrators. Anschließend wurden die Kennlinien für die Steuerspannung und die Ausgangsfrequenz aufgenommen, was essentielle Informationen für die Charakterisierung der Phasenregelschleife (PLL) lieferte.

Die Übertragungsfunktion des Voltage-Controlled Oscillators (VCO) wurde erfolgreich durch die erfassten Messdaten ermittelt. Der Phasendetektor wurde mittels Beispielmessungen überprüft, wodurch auch seine Übertragungsfunktion bestimmt wurde. Durch die Anwendung der Übertragungsfunktionen der PLL, des VCO und des Phasendetektors wurde schließlich die Übertragungsfunktion des Schleifenfilters ermittelt. Diese Ergebnisse bildeten die Grundlage für die präzise Dimensionierung der Bauteile des  PI-Filters.

Nachdem alle Komponenten individuell zusammengebaut wurden, erfolgte die eingehende Untersuchung des Einrastvorgangs der PLL. Die Ergebnisse zeigten eine reibungslose Funktionalität, die mit den Simulationen und theoretischen Erwartungen konsistent war. Insgesamt konnte der Versuch erfolgreich abgeschlossen werden, wodurch umfassende Einblicke in die Implementierung und Analyse einer PLL gewonnen wurden.