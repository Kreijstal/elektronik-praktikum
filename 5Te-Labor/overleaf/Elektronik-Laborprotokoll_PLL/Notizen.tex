\section{Notizen aus dem Labor}

Hybrid..bauen wir digital-analog
Oszillator--analog
Phasendetektor--digital

*Phasendetektor hat digitlae Zustände -1, 1 *


Eingangsspannung von maximal 5 Volt
Übertragungsfunktion davon
H_VCO ist eine Konstante K_vco = delta f/ delta U_e
*Je höher ist die eingangsspannng desto höher ist die Frequenz
*Man erwartert einen linearen Verlauf. Aber mit derm echten (sch..ßen) OPV ist es nicht immer so.
-->Protokoll--Bild, wie es 

Phasendetektor

D-FlipFlop
******
S-Pins auf Masse(Null) setzen. DAs benutzen wir nicht.
D-Pins ist bei V_DD
CLK ist für U_VCO
Resert reagiert NUR auf ddie flankende Steigende

Eingangssignale Kanal1 und Kanal2
--Protokoll einmal Bild für VCO schneller,  einmal Bİld für VCO langsamaer ein Bild für Ausgangssignal 0

Logik hat 3 Zustände  (1,-1,0) V_CC

H_PD=K_Phasendetektor (Konstane)=/elta U_A /Delta phasemax



VCO bauen 5Volt bitte nich 12Volt

Phaendetetkror ----Singly supply
(And Gatter)  (12Volt)


Schleifenfilter
Substrahierer ----- TL072 
Double Supply






oszilliert Kennlinie





\Notizen zur Durchführung

C0=Schutzkondensator VCc- Vcc+

R_1 18k
R_2= 27k

R_3=820ohm
C1=100nF Kondensatoeren

4ist OPV Schmitt 1 ist OPV Integrator

R3 zwischen 4OUT und 1IN-
R1  4IN+ 1OUT 
R2 4OUT 4IN+

C1 1OUT 1In-

gegeben: Der Basisvorwiderstand des Bipolartransistors. Nicht zu viel Strom in BAsis, trotzdem R6. Zu vile Strom alles kaputt. Zu wenig: Kollektro eMitter nicht groß genug. Eingangsspannung würde nicht gut weitergeleiete.

R4=460Ohm 

4OUT mit Mitte der Transistoren

NPN-Transistor
PNP Transistor

Mitte verbunden
--rechts unten in der Skizze-- invertierender Verstärker R5=R6= 1,2kOhm( für 1kOhm)

2OUt 2IN- R5
 3+    2IN-     R6
 3- und 3out kurzschluss




 4mal 1k ohm für phasendetektor


 0Volt 3Hz --ZLeckströme
 0,5Volt  303Hz newfile1
 1Volt 581Hz      
 1,5   847Hz
 2     1,09kHz
 2,5  1,31kHz  nwfile5
 3  1,51kHz
 3,5    1,72kHz
 4   1.92 kHz
 4,5 1.92kHz
 5   1,72kHz nwwfilw10


.csv

65
Bildoszilloskop

Wenn Phase mher voneinander abweicht, umso größer wird der PWM
Die Idee 

U_max für positive
10,7Volt...(e

für negative -7,4Volt


//Notizen 2.Teil



PI Filter

gerechnete Werte sind auf Papier, wie es zu rechnen war
* Wir hatten Gott sei Dank 1 mikro Farad

So folgt aus der rechnung: R1= 39kOhm
R2= 12 kOhm




100k benutzt
1,5k benutzt


PLL

Wgen debugging Zielen wurde das PI Filter umgetauscht
Siehe Bilder(WhatsAppGruppe)





Gelb ist U_ref
Blau PI Regler

Frequenz ehöhe, blau ist höher, 

Referenzssignal auf 0


Sprungantworrt

Wie sehen gelbe Signal das Rferenzsignal, der blaue Integrator steigt , Oszilliert selbst bisschen PI Filter, einraten ausrasten. Eİnges Ausrastvorgang

Sprungantrort als osr.
Butterworth Approximation, es gibt kaum Ğberschwingen. 

R2= tauschen 100kOhm zu 1 kOhm

Dämpfung soll geringer werrden
Überschwingen zu sehen,
Frequenz ausrechnen
(Evt. 

Sprungantwort(blaue) PI Fİlter plotten, das ist wichtig

die Spannung am PI Fİlter reagiert


Notizen zur Interpretation:

Was habt ihr aufgebeuat
Eİngangsgrößen

Ausgangsgrößen
Pllots

Bauteile Berechnungen


VCO augbeabut punktartig gemessen
Frequenz Wertetabelle, Kennlinie

Dimensionierung zu beschreiben
Dimensionierung ds SchmittTriggers
2/3VCO

Dimensionierung d des Integrators(?)

Ergebnisse
Wertetabelle
Richtige Kennlinie
Ideale Kennlinie

VCO muss simuliert werden
Vergleichen

Eine Regression an der Kennlinie (0-3,5)V

Die Koeffizienten VCO

Bild wo man Dreieck und Rechteck, als Beweis dass das VCO funktioniert


Das Ding integriert DC voltage
Dc rein integriert

Phasendetektor

Aufbau usw

Eingangsseitigg Funktionsgenerator eingegeben, Ergebnisse
Ein Bild, das Referenzsignal langsam, ein schneller

HPD KPD berechnen

Auch K_VCO Steilheit der Kennlinie, siehe Tafelbilder
Grob erkläre ,wie ein Phasendetektor funktionieren, bisschen Theorie

(Siehe Tafelbild in der 1. Woche)

Schleifenfilter, PI Filter aufgebaut, Rechnungen, 
D und w_n sind feste Werte
D für Butterworth Faches  Ordnung
Wn viel kleiner als 2000kHz damit wir kein Probleme bekommen

R_1 und R_2 berechnen, erkläre genau Schritte
Bodeplot analog 
Simulation dazu

Übertragungsfunkiton an der Tafel herleiten, wie man die Formeln für w_n T kommt von der Formel mMitte unten und oben 
Omega n und d erklären

PLL alles aufgebaut
;Eingangssprung im fUnktionsgenerator, gucken ob einrastet
Dimensionierung kommt vom vorherigen 
Eingeschwungenen Referenzsignal
Sprungantwort mithilfe des PI-Filters aufgebaut
Sprungantwort mit varieierten PI Filter
w_n ablesen können


%Erwartungshorizont
Was habt ihr aufgebeuat
Eİngangsgrößen

Ausgangsgrößen
Pllots

Bauteile Berechnungen


VCO augbeabut punktartig gemessen
Frequenz Wertetabelle, Kennlinie

Dimensionierung zu beschreiben
Dimensionierung ds SchmittTriggers
2/3VCO

Dimensionierung d des Integrators(?)

Ergebnisse
Wertetabelle
Richtige Kennlinie
Ideale Kennlinie

VCO muss simuliert werden
Vergleichen

Eine Regression an der Kennlinie (0-3,5)V

Die Koeffizienten VCO

Bild wo man Dreieck und Rechteck, als Beweis dass das VCO funktioniert


Das Ding integriert DC voltage
Dc rein integriert

Phasendetektor

Aufbau usw

Eingangsseitigg Funktionsgenerator eingegeben, Ergebnisse
Ein Bild, das Referenzsignal langsam, ein schneller

HPD KPD berechnen

Auch K_VCO Steilheit der Kennlinie, siehe Tafelbilder
Grob erkläre ,wie ein Phasendetektor funktionieren, bisschen Theorie

(Siehe Tafelbild in der 1. Woche)

Schleifenfilter, PI Filter aufgebaut, Rechnungen, 
D und w_n sind feste Werte
D für Butterworth Faches  Ordnung
Wn viel kleiner als 2000kHz damit wir kein Probleme bekommen

R_1 und R_2 berechnen, erkläre genau Schritte
Bodeplot analog 
Simulation dazu

Übertragungsfunkiton an der Tafel herleiten, wie man die Formeln für w_n T kommt von der Formel mMitte unten und oben 
Omega n und d erklären

PLL alles aufgebaut
;Eingangssprung im fUnktionsgenerator, gucken ob einrastet
Dimensionierung kommt vom vorherigen 
Eingeschwungenen Referenzsignal
Sprungantwort mithilfe des PI-Filters aufgebaut
Sprungantwort mit varieierten PI Filter
w_n ablesen können
